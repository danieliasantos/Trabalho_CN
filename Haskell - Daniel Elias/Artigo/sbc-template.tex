\documentclass[12pt]{article}
\usepackage{amsmath}
\usepackage{sbc-template}
\usepackage{multirow}
\usepackage{graphicx}
\usepackage{url}
\usepackage{hyperref}
\usepackage{verbatim}
\usepackage[brazil]{babel}  
\usepackage[utf8]{inputenc}
\usepackage[numbers]{natbib}

%\sloppy

\title{Uma análise de desempenho de processamento utilizando Haskell}

\author{Daniel Elias dos Santos\inst{1}}

\address{Departamento de Informática -- Instituto Federal de Educação,\\ 
	Ciência e Tecnologia de Minas Gerais (IFMG)\\ Sabará -- MG -- Brasil
	\email{danielias.santos@gmail.com}
}

\begin{document} 
	
	\maketitle
	
	\begin{resumo} 
		
	\end{resumo}

	\section{Objetivo}
	
	\section{Introdução}
		\subsection{Breve histórico}
			Haskell é uma linguagem de programação puramente funcional, polimórfica, estaticamente tipada e de avaliação tardia (\textit{lazy})\textsuperscript{\cite{haskellWiki}}. Seu nome foi dado em homenagem ao lógico Haskell Brooks Curry, cujo trabalho em lógica matemática serve como base para linguagens funcionais. A linguagem foi projetada por uma comissão, formada em 1987 a partir da conferência ``\textit{Functional Programming Languages and Computer Architecture}'' realizada em Portland, Oregon (EUA), para criação de uma linguagem funcional unificada, que estabelecesse uma base estável para o desenvolvimento de aplicações, e assim, popularizasse o paradigma funcional de programação\textsuperscript{\cite{histHas2007}}. A primeira versão da linguagem foi apresentada em 1990, e a primeira versão considerada estável foi apresentada em 1999. Haskell foi influenciada pelas linguagens Miranda e ML.
		\subsection{Características principais}
			\subsubsection{Tipos de dados}
			\subsubsection{Estruturas de dados}
			\subsubsection{Semântica}

	\section{Metodologia}
		\subsection{Ambiente}
		
	\section{Conclusão}

\begin{comment}	
	
	Texto com exemplo de referência a Tabela~\ref{tab:tarefas}.
	
	\begin{table}[ht]
	\centering
	\caption{Exemplo de tabela}
	\label{tab:tarefas}
	\begin{tabular}{ p{3cm}|p{6cm}|c }
	\multicolumn{3}{c}{Título do projeto} \\
	\hline
	Coluna 1 & Coluna 2 & Coluna 3\\
	\hline
	\multirow{3}{3cm}{Linha 1, coluna 1}  & Linha 1, coluna 2 & Linha 1, coluna 3\\
	& Linha 1.2, coluna 2 & Linha 1.2, coluna 3 \\
	& Linha 1.3, coluna 2 & Linha 1.3, coluna 3 \\
	\hline
	\multirow{3}{3cm}{Linha 2, coluna 1} & Linha 2, coluna 2 & Linha 2, coluna 3 \\
	& Linha 2.2, coluna 2 & Linha 2.2, coluna 3 \\
	& Linha 2.3, coluna 2 & Linha 2.3, coluna 3 \\
	\hline
	\end{tabular}
	\end{table}
	
	Texto da terceira subseção com exemplo de referência a Figura~\ref{fig:exemplo}.
	
	\begin{figure}[ht]
	\centering
	\includegraphics[width=.9\textwidth]{triangulo_quadrado_circulo.jpg}
	\caption{Figura de exemplo}
	\label{fig:exemplo}
	\end{figure}
	
\end{comment}
	\bibliographystyle{abnt-num}
	\bibliography{sbc-template}	
\end{document}
